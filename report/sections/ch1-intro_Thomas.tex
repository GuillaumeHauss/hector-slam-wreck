\subsection{How does this problem occur?}
%This list is purely for personal reference - Thomas
\begin{itemize}
	\item The need for exploration and expansion
	\item Technological advancements
	\item Curiosity
\end{itemize}

Exploration of unknown terrain and environments has been practised for many years. It originates from the human curiosity  and spirit to explore and gather information about the surroundings.

When dealing with unknown environments, unknown does not strictly refer to the environment itself but the person state of knowledge about the physics and composition of the environment. In a known environment, outcomes from every action can more or less be calculated or calculated. Where as in an unknown environment, it is a matter of investigation and figuring out what works and what does not, so that in the future a person can make better choice in that said environment.
%http://51lica.com/wp-content/uploads/2012/05/Artificial-Intelligence-A-Modern-Approach-3rd-Edition.pdf (page 44, section 2.3.2 Properties of task environments)

Even though most of the land has been explored and taken into consideration for its resources, the time will come where planetary and ocean exploration becomes a key factor for our technological advancements and our resources. Ocean exploration is important to provide data from deep-sea areas and will reduce the amount of unknown environments.
Gathering data and intelligence from the ocean also helps with managing the resources so that future generations can benefit from them. The ocean also provides information about future conditions and can help predict earthquakes and tsunamis. Investigating the deep-sea also reveals new ecosystems and possible sources for medication, food and energy. 
%http://oceanexplorer.noaa.gov/backmatter/whatisexploration.html

Humans have always had never-ending interest and need to push science and technology to its limits and then attempt to achieve something even further.  The many challenges humans have faced has led to many benefits for our society almost since creation. Space exploration helps further our understanding about the history of our universe and solar system. 
%http://www.nasa.gov/exploration/whyweexplore/why_we_explore_main.html

\subsection{Who is affected by this problem?}
%This list is purely for personal reference - Thomas
\begin{itemize}
	\item Humanity
	\item -- Some resources will be necessary in the future
	\item -- Expansion into space (Because of lack of space)
	\item Science and Scientists
	\item -- The never-ending quest for knowledge and information
	\item -- Technological advancements
\end{itemize}

In the future people .... 

Scientists are hindered by society because its becoming too focused on risks. Only 5 percent of the ocean has has been explored, being concerned about taking risks is what will put the future development in jeopardy. 
%http://www.nasa.gov/missions/solarsystem/Why_We_02.html 