\subsection{Where Is This a Problem (Unknown terrain)}

Unknown terrain can be said terrain which is not mapped. The unknown terrain we are interested in are other planets, and the botom of the ocean.
    
The bottom of the ocean is mostly sand, mud, and water [citation needed.]. This make it difficult to savigate with a rover, but relatively simple to navigate with a submarine. In water, sensors that use sound need to be calibrated to work with the correct pressure of the water surrounding it, which varies with the depth of the water. If the water is murky, light based sensors may also not work, depending on the wavelength of the light they use. Light also refracts in water depinding on the water density. It is possible to remote control vehicles in the bottom of the ocean.

Other planets can be much more complicated to navigate. The consistency of they atmosphere and terrain, may be not completely understood. This means that there is a higher chance of a rover being stuck on loose terrain, falling down sheer faces, and so on. It may also be the case that the planet has no atmosphere, so a drone would not be able to hover. One of the biggest issues with extraplanetary exploration is input lag. Other planets are very far away, this means that signals traveling to them can take several minutes at least and years at most to reach from the Earth. Therefore automation is better suited for this task. Other planets share the issues the bottom of the ocean has with sensors.

Disaster scenarios can also be said to be unkown terrain. Disaster scenarios means that the terrain might be loose or difficult to navigate, so a drone might be best suited for it. In most cases disaster scenarios happen in locations where the vehicle can be controlled manually
