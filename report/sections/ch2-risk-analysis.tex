\section{Risk management}
Possible risks that could occur before, during and after the implementation of the proposed bicycle-locking system, elaborated in this report, and presented in the form of a table together with a mitigation plan for each and every identified risk.

\subsection{Overview} 
We listed most of the risks that would influence the design process of our product. These risks will be used as criteria for the end-product.

Risks are considered on two different scales with 4 levels. As the legend under the table shows, it starts from Very High (VH) being the most severe to Low (L), which can be considered as a minor issue. The two scales are \textbf{Probability}, which represents the chance of occurrence, and \textbf{Impact}, which represents the severity of the risk in case it would actually appear.

\begin{table}[H]
	\begin{tabular}{ | p{4.5cm} | l | l | p{5cm} |}
		\hline
		\bfseries Risk factor & \bfseries Probability & \bfseries Impact & \bfseries Mitigation strategy \\ \hline
		 &  &  & \\
	\hline
	\end{tabular}
	\caption{Risk management and mitigation}
	\label{table:risktable}
	\caption*{
		\bfseries Legend:
		\begin{tabular}{l | l | l | l}
			VH - Very High & H - High & M - Medium & L - Low
		\end{tabular}	
	}
\end{table}
