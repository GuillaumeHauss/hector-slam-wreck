To complete this project successfully, different tools and processes were used. These are more \textit{commercial} compared to the ones in Chapter \ref{ch:methods-and-models}, \textit{Methods and Models}.

The full report of the P2 project was written using \LaTeX ~, just as the \textit{process analysis} and the project presentation. Alongside \LaTeX , we used a Git repository as our version control system for both the code and report, and we used GitHub as our platform-as-a-service for hosting and interfacing with this repository.

Meetings were scheduled using an Exchange calendar and set up a couple of days prior to taking place. During these meetings, a note-taker, whom we rotated on a weekly basis, would write down the outcome tasks and important discussion points. We also distinguished between strategy and work meetings, where the first one was more formal and required meeting minutes, while the latter was just to brainstorm and work on the project. Supervisor meetings were called on set schedule, but rather when it was necessary.

\textit{An example of these meeting minutes can be found in Appendix \ref{appendix:meeting-minutes-example}.}

After the meetings, we posted the minutes of the meeting on our GitHub wiki, and also went over our backlog and added our tasks and to-dos. The GitHub wiki served the purpose of storing meeting minutes, notes, but also our time and resource management, group contract and so on. 

Communication outside of the group was done mainly on Facebook. When we could not be physically present for our meetings, we held them over a Skype call.

Peer-reviewing sessions were scheduled after each major deadline, to discuss the work that each individual had written and to ensure that the quality of the report lived up to expectations.

To produce the codebase which we wrote, the Arduino IDE and other terminal based text editors, like \textit{nano} and \textit{vim} were used.

To create UML diagrams, we used a tool called Visual Paradigm, that can generate all kinds of charts and diagrams based on the proper UML standard. To create flow charts and other figures, we used Google Draw and Gliffy, two online-based image editor and vector graphic tool.

To keep track of our workflow we used an online Kanban board, where group members encouraged each other to keep it updated with the individual user stories and tasks. Deadlines and milestones were set to ensure that we were on track, while a Gantt diagram was used to visualize this.

\section{Discussion and Conclusion}
Out of the tools listed above, we found Git, the version control system and GitHub, the PaaS tool to be the most useful of them all. Having a decentralized version control system has saved many hours of trial-and-error, when trying to fix code or writing the report. 

Using the terminal text editors, it helped us to brush up on our Linux skills, but also due to their raw format, made it easier to work on them simultaneously. 

The Exchange calendar was very useful to keep track of the upcoming meetings and reminding ourselves of meeting times. Formal invites should be sent out before each meeting.

\LaTeX ~was also extensively used during this project, and therefore our levels of using it has improved since the P1. We certainly hit less walls and experienced less problems using it, and therefore it served as a very effective tools to produce the written reports.

Our online-based Kanban board, Kanbanflow.com, was very effective in the beginning and useful to keep track of the tasks, but just as the development phase kicked in, we started spending more time together in the group room and therefore we started neglecting it, due to it being inefficient to update an online tool while all of us being physically present in the same room. The lack of push-notifications and mobile applications for the Kanban board would also have been more useful, as of now, we always need to remind ourselves to visit the website and update it. For our next project we need to re-evaluate the need of an online Kanban board, and find a better alternative that might be just a to-do list application with mobile extension.

Even though, the meeting minutes were very time-consuming to write and put together, sometimes they were very useful, for example when someone missed a meeting and had to catch up on the tasks and meeting points. Therefore we think we should re-evaluate in the future whether writing the meeting minutes is important or not.

The GitHub Wiki seemed to be a very good tool to use, as we stored many of the out-of-category papers and notes in there. Since our last project, we certainly started using the wiki more often, but there is still a lot of room for improvement, as we still put many of the files and notes into our Facebook group. For future collaboration, we thought about structuring our wiki early on the project and force ourselves to use it more, by putting all the files in one shared space and link them on the wiki. We also used our wiki as a place where we collected links.
