During the first weeks of the project work, we all signed a group contract to ensure we all lived up to each others expectations. This contract can be seen in \textit{Appendix \ref{appendix:group-contract}}. The supervisors also requested a contract, to work as agreement of expectations. An additional deadline contract was then created later during project, which will be discussed during the \textit{Group Work} section more thoroughly, Chapter \ref{ch:group-work}.

Everything that took place during our strategy meetings were noted down and added to our GitHub wiki, whic can be found in \textit{Appendix \ref{appendix:github}}. Pictures of prototypes, observations or of the blackboard were all added to the Facebook group, to ensure the information was accessible at all times.

For the electrical parts of this project, we did a lot of handwritten notes and sketches in small notebooks with the different setups and connection, but we did less on documenting the software side.

The testing phase was documented using a mixture of pictures, screenshots and video recordings of our different experiments, which was then converted into the testing chapter of our report. 

\section{Discussion and Conclusion}

We changed very little from our previous group contract, formalized in the P1 project, except that we added few rules about the usage of new tools like Kanbanflow. In this project, we continued using the same documentation processes, as in the P1 project, since it worked well before. 

We decided to put much less into our Facebook group then before and upload more of the documents and notes to the Github wiki. This is so things will not get lost in the overflow of information that has been happening on Facebook. The initiative was good, but we did not follow this through and is clearly something we will work on for the next project.

The use of a engineering notebook was much less in this project, and the reason for that was that our project was had a much bigger scope software-wise than physical-. Because the changes in software are much frequent and done on the spot, there was no reason to take them out and write them to a physical notebook. All the code was uploaded frequently to our Git repository and that keeps a good track of it in time of development. For the electrical development, we noted down all changes and pin-outs.

The documentation of testing was done really well and planned ahead. Previously, we lacked the documentation part of our development, and therefore it served as a big help for us in writing a good and extended testing chapter this time.