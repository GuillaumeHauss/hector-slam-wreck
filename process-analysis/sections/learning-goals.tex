The main goal of our project was to create a two dimensional map of an unknown environment and autonomously navigate through it, while also fulfilling the learning goals stated in the curriculum for the project.
\textit{Through theoretical and practical work on a selected problem, the
students acquire knowledge in the electronics and computer engineering
discipline, as well as use appropriate methods to document that the
problem has a relevant social context. The problem is analysed by
decomposition into sub problems in order to formulate a technical
problem that can be solved by using analog electronic systems that
interact with the environment in one way or another. The complete
solution is assessed with respect to the relevant social context. Compared
to the first semester, this semester focuses more on the continuous-time
(analog) aspects of electronic systems as well as interaction with the
surroundings in greater detail.}

At the start of the project and before the status seminar we discussed our personal learning goals, but also the ones set by the university.  Each member was responsible for achieving their own personal learning goals and in collaboration we aimed to achieve the learning goals set by the university.\\

The group had a few common learnings goals which included learning how to apply analog sensors, knowledge about 2D/3D mapping, improving group collaboration and how to use CAD software.
\section{Discussion and Conclusion}

%Inactivity before development

%our scope was too big. it did not match our learning goals
% We spent too much time delving too deep into theory and not enough investigating tools
% Even without issues, we would not have had time to combine our prototypes
