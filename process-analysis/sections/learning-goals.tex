The main goal of our project was to create a two dimensional map of an unknown environment and autonomously navigate through it, while also fulfilling the learning goals stated in the curriculum for the project.
\textit{Through theoretical and practical work on a selected problem, the
students acquire knowledge in the electronics and computer engineering
discipline, as well as use appropriate methods to document that the
problem has a relevant social context. The problem is analysed by
decomposition into sub problems in order to formulate a technical
problem that can be solved by using analog electronic systems that
interact with the environment in one way or another. The complete
solution is assessed with respect to the relevant social context. Compared
to the first semester, this semester focuses more on the continuous-time
(analog) aspects of electronic systems as well as interaction with the
surroundings in greater detail.}

At the start of the project and before the status seminar we discussed our personal learning goals, but also the ones set by the university. Each member was responsible for achieving their own personal learning goals and in collaboration we aimed to achieve the learning goals set by the university.

The group had a few common learnings goals which included learning how to apply analog sensors, knowledge about 2D/3D mapping, improving group collaboration and how to use CAD software.

\section{Discussion and Conclusion}

Getting our learning goals materialized was still hard task to do, up until we had our status seminar, where we got some guidance from the supervisors. Personal learning goals should reflect on one student's ambition on what he or she would like to learn during a set semester and to what extent should it deepen its knowledge on the set topic. 

Initially, we set our learning goals too broad, which could have backfired on us by the end of the semester, as we haven't concretized the level of understanding that we would like to achieve, and therefore could have been questioned on levels we never tried to understand. As this was pointed out to us during the status seminar, all of our learning goals were changed and made more concrete. 

Setting our personal learning goals around the project seemed to be logical, but having set our learning goals first, might have saved us some time to find the right project and topic to work on. Coming up with the learning goals before concretizing our project, could help us to get a project we all have a high level of interest in, instead of formulating the project around our learning goals and not the other way around, therefore making the process more efficient. Other aspects should be taken into consideration when coming up with our learning goals, like how relevant it is to our semester learning goals and to our study.

As we formulated our learning goals around a project we had already set, our scope was way over our understanding and time-frame. That being said, it was also quite a new topic and one that required a large amount of research for understanding. Even without all the arising issues we had during development, we would not have had enough time to finalize our set prototype requirements and to combine our separate parts that we developed.

\textit{Our personal learning goals can be found in Appendix \ref{appendix:personal-learning-goals}, while the curriculum learning goals in Appendix \ref{appendix:curriculum-learning-goals}.}